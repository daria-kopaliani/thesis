\documentclass{article}\usepackage[]{graphicx}\usepackage[]{color}
%% maxwidth is the original width if it is less than linewidth
%% otherwise use linewidth (to make sure the graphics do not exceed the margin)
\makeatletter
\def\maxwidth{ %
  \ifdim\Gin@nat@width>\linewidth
    \linewidth
  \else
    \Gin@nat@width
  \fi
}
\makeatother

\definecolor{fgcolor}{rgb}{0.345, 0.345, 0.345}
\newcommand{\hlnum}[1]{\textcolor[rgb]{0.686,0.059,0.569}{#1}}%
\newcommand{\hlstr}[1]{\textcolor[rgb]{0.192,0.494,0.8}{#1}}%
\newcommand{\hlcom}[1]{\textcolor[rgb]{0.678,0.584,0.686}{\textit{#1}}}%
\newcommand{\hlopt}[1]{\textcolor[rgb]{0,0,0}{#1}}%
\newcommand{\hlstd}[1]{\textcolor[rgb]{0.345,0.345,0.345}{#1}}%
\newcommand{\hlkwa}[1]{\textcolor[rgb]{0.161,0.373,0.58}{\textbf{#1}}}%
\newcommand{\hlkwb}[1]{\textcolor[rgb]{0.69,0.353,0.396}{#1}}%
\newcommand{\hlkwc}[1]{\textcolor[rgb]{0.333,0.667,0.333}{#1}}%
\newcommand{\hlkwd}[1]{\textcolor[rgb]{0.737,0.353,0.396}{\textbf{#1}}}%

\usepackage{framed}
\makeatletter
\newenvironment{kframe}{%
 \def\at@end@of@kframe{}%
 \ifinner\ifhmode%
  \def\at@end@of@kframe{\end{minipage}}%
  \begin{minipage}{\columnwidth}%
 \fi\fi%
 \def\FrameCommand##1{\hskip\@totalleftmargin \hskip-\fboxsep
 \colorbox{shadecolor}{##1}\hskip-\fboxsep
     % There is no \\@totalrightmargin, so:
     \hskip-\linewidth \hskip-\@totalleftmargin \hskip\columnwidth}%
 \MakeFramed {\advance\hsize-\width
   \@totalleftmargin\z@ \linewidth\hsize
   \@setminipage}}%
 {\par\unskip\endMakeFramed%
 \at@end@of@kframe}
\makeatother

\definecolor{shadecolor}{rgb}{.97, .97, .97}
\definecolor{messagecolor}{rgb}{0, 0, 0}
\definecolor{warningcolor}{rgb}{1, 0, 1}
\definecolor{errorcolor}{rgb}{1, 0, 0}
\newenvironment{knitrout}{}{} % an empty environment to be redefined in TeX

\usepackage{alltt} 
\usepackage{multirow}
\usepackage{booktabs}
\usepackage{tabularx}
\IfFileExists{upquote.sty}{\usepackage{upquote}}{}
\begin{document}

% Приклад посилання на таблицю
У табл.~\ref{tab:ellipse.hyperbola.parabola} наведені деякі
формули для еліпса, гіперболи і параболи.

% Приклад таблиці
\begin{table}[htbp]
\caption{Еліпс, гіпербола і парабола. Деякі формули}
\label{tab:ellipse.hyperbola.parabola}
\begin{tabularx}{\textwidth}{|X|c|c|c|}
\hline
                   & Еліпс                                    & Гіпербола                                & Парабола          \\
\hline
Канонічне рівняння & $\frac{x^2}{a^2}+\frac{y^2}{b^2}=1$      & $\frac{x^2}{a^2}-\frac{y^2}{b^2}=1$      & $y^2=2px$         \\
Ексцентриситет     & $\varepsilon=\sqrt{1-\frac{b^2}{a^2}}<1$ & $\varepsilon=\sqrt{1+\frac{b^2}{a^2}}>1$ & $\varepsilon=1$   \\
Фокуси             & $(a\varepsilon,0)$, $(-a\varepsilon,0)$  & $(a\varepsilon,0)$, $(-a\varepsilon,0)$  & $(\frac{p}{2},0)$ \\
\hline
\multicolumn{4}{|l|}{Корн~Г., Корн~Т. Справочник по математике. М., 1974. С.~72.} \\
\hline
\end{tabularx}
\end{table}


\end{document}
